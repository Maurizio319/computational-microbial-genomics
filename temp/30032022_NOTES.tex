%\graphicspath{{chapters/images/01/}}

\chapter{Whole-genome epidemiology characterisation, and phylogenetic reconstruction of \textit{Staphylococcus aureus} strains in a paediatric hospital}

\section{Introduction}
G+, facultative anaerobe, widely distributed in the environment. it is a common colonizer. main player in poisoning. Several serious diseases.  very bad skin infections. bacteremies can be lethal. 
Tricky to treat, resistance to antibiotics. two main roots, that S. aureus can take. Adhesins, attract the host tissue and get covered, the immuno system doesn't understand the bacteria changing self. toxins to kill immune cells adaptive, the immunoglobulin binding protein immobilize the antibodies. proteases cleave the 
kill colonizers,  triggers the inflammation (redness)

after few rounds replication pathogens escape the neutrophils and escape the cells. 

Antibiotic resistance, penicillin was greately used, generated penicillin resistance. 
isolates of resistant in UK. The world was populated by resistant populations. methicillin is not used anymore. 
S.aureus resistanto to beta-lactams, it is able to acquire other resistances. 

How methicillin resistance works?
loss of cell wall cross-linking, cells with defective cell walls. 

sequence to area sorrounding the region containing mecA,wich is a chromosome,

S. aureus not well known, it causes high morbility (infections), high mortality rate, diffused among hospitalized patients. In 2017 the OHO inserted the S. aureus as an high priority mo. About 5000000 deaths were associated to S. aureus resistance in 2019.

Great underestimation of MSSA. 


\section{Work explanation}
pipeline

\begin{enumerate}
	\item \textbf{Wet lab procedures}: hospital staff, selectiv media to grow staphilococcus aureus, antibiotic usage, extraction of DNA from the colonies of staphylococcus aureus. 
	\item \textbf{Whole gennome sequencing} of the isolates
	\item \textbf{quality control} of the reads
	\item \textbf{assembly} (SPAdes)
	\item \textbf{Genes annotation} (PROKKA)
	\item Geenomes allignement core genes, accessory genes, alligned genes (to find specific point mutations)
	\item phylogenetic analysis on core genes, single genes,
	\item genomic analysis to findvirulence, resistance, antigens, check for the presence of an outbreak.  
\end{enumerate}

multilocus sequence typing (MLST)

cgaracterusubg isolates by sequencing
With genomic analysis

SCCmec, it is a mobile genetic element, the mec gneecomplex (mecA + mecI and mecR)

junk regions were renamed junction regions


675 PCR

PVL is used as an indicator of S.aureus virulence. it is encoded in a prophage. 1464 core genes. Presence of virulent genes. 27% so quite variant

41 spa-types, spa typing was limited for staphylococcus aureus. 

accessory genes are diverse between them

52\% not assigned, five of the 

ST228 is namd the south german/italian epidemic clone. 
some STs are associated to livestocks, bovine mastitis. 
ST395 exchanges DNA  with CoNS because of modified wall teichoic acid (WTA), so resistances could be due to this tranfers. 

culture strategies and colors can be used to detect the 

cystic fybrosis (difficulty in cleaning the lungs from mucus, ). chronic infections are less virulent
if we look to skin lesions, 

cassette type 4


diversity of virulence factors and antigens. you can thee the presence/absence of each gene, some genes are always present. The resistance to vancomycin is laways not present (last resort antibiotic). Resistance genes in chronic infections, virulence genes in acute infections. only in one sample.

No vaccine to prevent staphylococcus aureus, having a vaccine would help.

vaccines coulf be produced by highly prevalent genes, but high deree of variabiliry/indels. mntC is present in all of the positive isolates.  

whole genome sequencing can help in better understanding the pathogens, check variation in genes of interest, observe emerging of unexpected clones. 