Especially important in case sequencing for the first time an organism, or so distant that reference genomes are not usable.

\\

Assembly algorithms are based on this prpocess: identify overlaps on the reads (look at the overlaps and ideally I end up with the total result). Keep track of all the potential overlaps. In graph reads are connected through overlaps. It has to be identified a path. There are several algorithms available, find a mathematical solution to get a result out. The genome is usually neger a single contig, but we generally finish with several contigs .

\\

\textbf{The general pipeline}:
\begin{enumerate}
\item find overlapping reads
\item merge some good reads into longer contigs
\item it is possible to link contigs to form scaffolds
\end{enumerate}

\\

\textbf{Contig:} A contiguous sequence formed bt several overlapping reads with no gaps.
\textbf{Scaffold:} Time consuming operation, it cana take really a lot of work
\textbf{Consensus sequence:} A set of possi...

\\

\subsection{Feasibility of sequence assembly}
Doing the perfect assembly is impossible techncally, contrarily to mapping. Error rate coverage and read length are the most important parameters. 5x 4x coverage.




\textbf{Merging overlapping reads}
Find a score for a match, as in mapping. mismatch is due to error noise. Genomes can be diploid...

Graphs can be used for asssemblies. The vertixes (nodes) are the reads and each of them are connected between them. Graphs are ideally cyclic, as most of the mo have circular DNA. 


great reads generate greater error rates. balance between error rate and length of hte reads. 


\\

.... (TODO to be added) ...
\\

Overlap graphs: each edge indicateds an overlap.
Find a path to touch all nodes but only once, Hamiltonian paths. You want the path that maximizes overlaps (in an ideal situation). ALtough, the problem is too computationally costly. 

\\

\textbf{A greedy approach:} choose the local best solutions \\

Merging consecutive nodes, dead ends can be removed


several problem overlap graphs
\begin{enumerate}
\item repeats
\item it is not tractable
\end{enumerate}\\

de Brujin graphs (DBG)


de Brujin graphs: a bit complex, all the assemblers based on it. k-mers are the nodes, they are taken from the input. Solutions are based on Eulerian paths: visit edges only one time. The advantage is that we have not a numerous amount of cases as in the case of Hamiltonian.

...

N50 represent the median length of the contigs. N50 has to be the length of a real contig. 